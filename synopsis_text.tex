\documentclass[a4paper,12pt]{article}
\usepackage[a4paper,margin=1.0in]{geometry}
\usepackage[latin1]{inputenc}
\usepackage{amssymb}
\usepackage{palatino}
\usepackage{color}
\usepackage{graphicx}
\usepackage{epstopdf}
\usepackage{subcaption}
\usepackage[final]{pdfpages}

\begin{document}
	
	
\includepdf[pages=-]{soubhik_synopsis_cover_page.pdf}



\title{Time-Resolved Study of Intense Laser Cluster Interaction}
\author{Soubhik Sarkar}
\maketitle
\sloppy
\pagenumbering{roman}


\begin{abstract}
This thesis attempts to record the experiments performed on nano-metric clusters subjected to intense light fields with the aim to shed light on the fast dynamics and evolution of such systems.
\end{abstract}

\section{Introduction}
Interaction of intense laser pulses with nano-metric clusters has been a subject of considerable interest in the field of intense laser plasma science for both fundamental and applied reasons. The sub-wavelength dimension of clusters and the lack of usual modes of energy dissipation of bulk condensed media ensures unrestricted energy deposition into the clusters and efficient re-partitioning of the laser energy into the reaction products of high energy particles and radiation. The combination of these properties results not only in the production of high energy particles like MeV ions\cite{ionditmire} and neutrals\cite{neutralrajeev} and KeV electrons\cite{hotelectronjha,electronassymetryvk} and short bursts of x rays\cite{riju} but also allows for the clean study of the reaction products formed in such interactions. The interaction of ultrashort laser pulses with the nanometer scale clusters and the evolution of the cluster plasma over a period of 100 fs to about a few ps is the point of focus of this thesis. 
The general evolution and dynamics of the laser driven cluster plasma can be broadly broken down into the following steps\cite{revkrainov,revsaalmann,revfennel}. 
\begin{enumerate}
    \item The ionization of the cluster atoms by field driven tunnel and over-barrier ionization processes initiating the formation of an over critical density plasma medium.
    \item The free electrons being subsequently driven in the laser field are heated though resonant and inverse Bremsstrahlung processes and causing further ionization leading to strong charging of the cluster.
    \item The highly charged core of cluster ions left behind, respond by expanding under the strong thermal and Coulomb pressure but on a slower time scale of few ps owing to their inertia, resulting in the final stage of the interaction; the hydrodynamic expansion and Coulomb explosion.
\end{enumerate}
Though the general sequence of steps of the interaction is reasonably well accepted, the details of the interaction are often not fully understood. The numerous studies on efficient energy absorption by clusters and the consequent production of high energy ions, electrons, neutrals and radiation have attributed the observations to resonant absorption mechanism and cluster expansion and explosion. Resonant absorption mechanism has indeed served as a reliable means to explain the large energy deposition in clusters interacting with femtosecond pulses of $\gtrapprox$ 100 fs\cite{explosionditmire}. Observation and occurrence of resonant absorption mechanism, deriving from the collective dielectic properties is thus often referred to as a point of distinction between clustered and un-clustered atoms. But how the nature of interactions and energy deposition mechanisms change with systematic transition to shorter laser pulse duration is an open question. Should resonance absorption be always assumed to explain increased energy deposition in clusters? Or how important still are collective mechanisms for interaction of clusters with shorter pulses (25 fs), like those studied in this thesis? Similarly, our knowledge about the details of the cluster expansion and explosion phenomenon are not complete. Cluster expansion, the generally accepted final stage of the laser-cluster interaction is what converts the potential and thermal energy of the charged cluster into the final kinetic energy of the high energy reaction products. But, the clusters in their course of interaction are subjected to forces both preferentially directional and radial and studying the evolution of the cluster plasma under their combined influence could throw more light onto the subject of cluster expansion and explosion itself. 

Since the entire dynamics in femtosecond laser cluster interaction plays out within only a few ps, we performed time-resolved experiments to address questions to the like raised above. We employed variations of the standard pump-probe technique, among other experiments to study the time evolution of the cluster plasma.
We also performed time resolved experiments to determine whether it was possible to generate in cluster plasma, strong magnetic fields as is generally associated with laser plasma interaction. We however were interested in generating and detecting axial magnetic fields due to the effects of Faraday rotation using a circularly polarized pump and a linearly polarized probe. 

In the following sections, each of the major experiments are described in some details. 

We have performed absorption experiments with varying pulse durations and found that the essential distinction between the behaviour of clustered and unclustered gas disappears for pulses $\lessapprox$ 100 fs. Systemetic measurement of absorption can be used to infer the type of dominant processes and has been used here as a simple and effective tool to prove that that for very short pulses collective mechanisms of nano-plasmas plays only a very insignificant role in the interactions. Similarly by measuring absorption in pump-probe experiments and modelling of the cluster plasma as an evolving dielectric we were able to infer about the effective shape of the evolving plasma system. In the context of the generation and detection of axial magnetic fields, however, our observations indiacte the presence only of comparatively weak magnetic fields, spatially averaged to no more than 1 T over the entire interaction region.

\section{The Cluster Target}
Nanometric aggregates of atoms or molecules held together by inter-atomic Van der Waals forces and referred to as clusters, was the only exclusive class of targets primarily used for all experiments described in this thesis. The motivation for choosing this class of target for investigating laser-plasma interaction has already been alluded to in the introduction.
In the experiments to be described we have used only clusters formed from atoms and molecules which are otherwise in the gaseous state at S.T.P. namely N$_2$ and Ar.
Clusters of otherwise gaseous species are normally produced in a 'supersonic jet expansion' inside a vacuum chamber. The technique involves holding the gas at high pressures and then letting it expand into vacuum through specially designed nozzles which allows the gas jet to reach supersonic speeds upon exiting the nozzle. This in turn leads to adiabatic cooling decreasing the transverse temperature in the jet and ultimately leading to condensation of individual atoms into clusters, albeit depending on the polarizability of the expanding species.

Apart from the nature of the gas itself, several other factors like the stagnation conditions(temperature and pressure) and the geometry of the nozzle all influence the process of cluster formation. For all practical purposes, the estimates of average cluster size in this thesis were made from Hagena's scaling law\cite{clustersourceHagena}. Hagena's empirical scaling laws allow one to estimate the average cluster size given the stagnation condition, nozzle geometry and gas properties. We had, as an exercise, in order to characterize the cluster beam and also to check that the pulse-valve and nozzle used for cluster production in our experiments were performing as expected with the variation of the controlling parameters, performed Rayleigh scattering experiments(to be described later) on the cluster jet and found a consistent performance of our cluster source so as to be reliably used in the following experiments. Also as has been experimentally observed\cite{gspann}, the distribution of cluster sizes in the cluster beam consistently follows a log-normal distribution. But since performing experiments to measure the cluster size distribution is itself technically involved and challenging, the knowledge about the size distribution was but simply assumed wherever required.

The cluster generating source used for the experiments here was a Parker solenoid pulse valve driven by Parker pulse valve driver. The pulse valve operates by releasing the valve nozzle by drawing out a poppet every time a solenoid is energized by a pulsed current, supplied by the pulse valve driver. In the valve closed condition however, the poppet is held in place against the nozzle by a spring mechanism that is able to maintain a very low leak rate even at a pressure difference of few 10s of bars across the nozzle. 

\section{Eperiments}
\subsection{Source Characterization by Rayleigh Scattering}
In order to check the performance and reliability of the cluster generating pulse valve, we had performed Rayleigh scattering experiments establishing the source performance to be reliable and consistent with expectations with the variations of controlling parameters, specifically the stagnation pressure. 
Scattering from clusters of sub-wavelength dimensions can be suitably described by Rayleigh scattering of the incident radiation, treating individual scatterers as singular dipoles without any spatial field variation over its physical dimension. Given that the Rayleigh scattering cross-section $\sigma_{RS}$ for scattering at any angle on a plane perpendicular to the polarization of the incident radiation is $\propto$ \(\frac{r^6}{\lambda^4}\), where r is the cluster radius and $\lambda$ is the incident radiation, the measured scattered intensity S$_{RS}$ which is proportional to the product of the cluster density $n_{c}$ and the square of the average cluster size $N_{c}$, can be used to determine the variation of the cluster sizes with the different controlling parameters and compare the results with those predicted from Hagena's scaling laws. 

Using a Thorlabs PMT100A photo-multiplier tube, we measured the photons scattered from a low intensity(below ionization threshhold) 532 nm nanosecond pulsed laser beam by the cluster jet at right angles to both the beam propagation direction and the direction of polarization of the incident beam. The variation of this signal can be directly related to the variation of the average cluster size with change in the various controlling parameters like the stagnation pressure, pulse valve open time etc. This exercise not only helped to establish confidence in the performance of the pulse valve but also figure out the range of the operating parameters over which such confidence of a steady operation can be assumed. 

\subsection{On the (Un)-Importance of Collective Processes in Cluster Nano-Plasma}

Owing the size-limited nature and minuscule dimension of the target, the cluster nano-plasma are expected to exhibit strong collective processes due to the influence of electromagnetic fields, both external and induced. As an figure of merit, assuming even a very moderate temperature of 100 eV and a modest charge density of $\sim 10^{21}$ per cc results in a Debye length of $\approx$ 2.5 nm which is itself a significant fraction of the few nm dimension of the clusters. Collective phenomenon in clusters are a key to understanding the strong coupling of light energy into the nano-plasma. Upon the initial ionization of the near solid density clusters, the charge density can reach upwards $\sim 10^{21}$ electrons per cc. These ionized electrons, still quasi bound to the overall cluster potential, but now driven by the oscillating laser field sets the stage for further ionization through collisions and internal field enhancements due to charge separation. This scenerio of collective electronic oscillation against the background of ions can be simplistically modelled as a driven oscillator, with its characteristic natural frequency $\omega_{pe}$, the plasma frequency and driven by the laser field. Though the dimensions of the clusters are small enough to neglect any spatial variation of the external field over the volume of the clusters itself yet, the capacity of the plasma to sustain a charge separation to resist the field penetration, results in the internal field being different from the applied laser field E$_{0}$. This internal field, E$_{in}$ can be expressed in terms of the complex dielectric constant of the cluster plasma.
\begin{equation}\label{eqn1}
E_{in}=\frac{3}{|\epsilon + i\nu|}E_{0}
\end{equation}
where \({\epsilon}\) and \({\nu}\) are the plasma dielectric constant and the electron-ion collision frequency respectively. This internal field is therefore dependent on the collective property of the cluster plasma, more specifically, on the dielectric constant, which is itself a function of the cluster charge density. Now, since the cluster density is itself a time varying entity, evolving as the cluster expands from its initial solid like density, under both internal thermal and Coulomb pressures, the complex dielectric constant and as a result, the internal field itself, evolves in a correlated manner with the clusters themselves. These driven system of dipoles, as usual, naturally exhibit a resonance and it occurs when the internal field passes through its maxima at some point, as the charge density decreases overtime. This phenomena is quite central to understanding the deposition of energy into the clusters because the increased internal field also translates to increased energy deposition which manifests as increased absorption, higher charge states and emission of x-rays\cite{xrayissac,zweibeck}. Observations indicating an increased deposition of energy at an optimal time interval after the onset of photo-ionization by the laser pulse, has been repeatedly confirmed in numerous time resolved experiments. This resonance, that occurs, leading to an observable increase in the energy deposition in the cluster medium, has come to pass as a defining trait of clusters interacting with ultra short laser beams. 

This resonance is found to more or less occur at $\sim10^{2}$ fs and decrease monotonically either way, as one moves away from the point of resonance, in time. But, it should also be kept in mind that it is not only the collective mechanisms per se that are alone important in the description of laser-cluster interaction; field driven processes affecting the atoms individually also plays a part, and expectedly more so in the initial stages of charging of the clusters. And so, whether the strength of this collective response is always dominant over the other contributing factors, as is often assumed, should be a relevant and important question in this context.

\begin{figure}[h]
  \centering
  \begin{minipage}[b]{0.495\textwidth}
    \includegraphics[width=\textwidth]{grating_exp.eps}
    \caption{Experimental setup to study the variation in absorption with change in pulse duration; the grating pair was used to vary the pulse duration.}
    \label{grating_exp}
  \end{minipage}
  \hfill
  \begin{minipage}[b]{0.495\textwidth}
    \includegraphics[width=\textwidth]{Ar_N2.eps}
    \caption{Comprison of absorption in Ar and N$_{2}$; a stark difference in well-clustered and un-clustered systems.}
    \label{Ar_N2}
  \end{minipage}
\end{figure}

To address this, we performed an experiment to measure the absorption of laser energy as the laser beam traverses the cluster jet. The schematic is provided in Fig.\ref{grating_exp}. As indicated in the schematic, the pulsed laser beam was focused onto the cluster jet and the energy of the transmitted beam, as measured by the photo-diode placed behind the point of interaction, was compared with that of the original incident beam to arrive at an estimate of the energy absorbed by the cluster jet. Given the KHz repetition rate of the incident laser beam, it was possible to arrive at a statistically consistent value of this absorption at every data point by collecting the data for a practically achievable length of time. 
The laser pulses, each containing $\approx$ 3.5 mJ of energy and spectrally centered around 800 nm, with a bandwidth of $\approx$ 60 nm produces transform-limited pulses of 25 fs. This is achieved by optimizing the pair of transmission gratings comprising the final, compression stage of the laser unit. However, by changing the grating separation distance from its optimized value, one can add dispersion and produce a chirped output beam with longer pulse duration. This is the handle that was used to produce longer pulses which were in turn used in the experiments, to study how the absorption for longer pulses compare with those as short as the transform-limited pulses of pulse duration $\approx$ 25 fs. Also, in order to test the effect of the clusters itself, two gases, Ar and N$_{2}$ with largely differing ability to form clusters were compared in the process. According to the Hagena's scaling laws, under the same conditions, the average size N$_{c}$ of clusters for different gases is dependent on a constant K, defined by the gas properties. The dependence is captured through the Hagena parameter $\Gamma$$^{*}$ as
\begin{equation}\label{eqn2}
N_{c} \sim (\Gamma^{*})^{2.35} \sim K^{2.35}
\end{equation}
The value of this constant K, for N$_{2}$ and Ar is 528 and 1650 respectively. Also, the other important reason for the choice of these two particular gases was that, although their ability to condense into clusters are largely different, as indicated by the constant K, their ionization potentials are quite similar. For example, the first two ionization potentials for Ar are 15.75 eV and 27.69 eV and, 15.58 eV and 27.9 eV for N$_{2}$\cite{crc}.

The result of the experiment is summarised in Fig.\ref{Ar_N2}. The larger clusters of Ar exhibits a clear resonance resulting in increased deposition of energy at pulse duration around a few 100 fs. It is so because, the effective resonant deposition of energy can only occur for optimally long pulses which peaks around the time, the expanding clusters reach the critical density. The exact value of the pulse duration at which the peak occurs and the range of pulse duration over which this increased absorption is manifested depends on the stagnation pressure of the gas behind the pulse valve or, conversely, the average cluster size. However, for shorter pulse duration ($\lessapprox$ 100 fs), the decreasing trend of the absorbed energy fraction is suddenly altered and it begins to increase monotonically again as the pulse duration is brought closer to the shortest possible pulse of 25 fs. On the other hand, N$_{2}$ shows a monotonic decrease in the absorption of laser energy as the pulse duration is increased from 25 fs. The absence of any significant signature of the resonance as observed for Ar stands to confirm that the gas jet of N$_{2}$ is primarily constituted of un-clustered gas and/or much smaller clusters - at best. The absorption and its variation with pulse duration for N$_{2}$, can be thus argued to be dominated by the field driven single particle processes affecting individual un-aggregated atoms in the laser focal volume. This monotonous decrease in absorption can be attributed to the decrease in laser intensity with increasing pulse duration. We have modelled the absorption of light by N$_{2}$ molecules dispersed over the focal volume of the laser, assuming a realistic number density of the gas molecules and the measured focal waist and also taking into account the intensity variation over the focal volume. The energy absorbed, though, was calculated by considering only the field driven single particle processes like tunnelling and over-barrier ionization and collisional ionization by field driven electrons. This calculated curve was found to closely follow the one obtained experimentally and shown in Fig.\ref{Ar_N2}. This further helps to establish the dominant role of these field driven single particle processes for the poorly clustered system of N$_{2}$ molecules over the entire range of the pulse duration studied. However, from Fig.\ref{Ar_N2} it is also apparent that the absorption in Ar also follows a trend similar to that of N$_{2}$ till about 100 fs, after which point the resonance feature becomes the dominant characteristics of the absorption curve. This dominance and exhibition of single particle effects, even for the clusters of Ar for pulses $\lessapprox$ 100 fs was further established by comparing for the two gases, the variation of absorption against the stagnation pressure, with the pulse duration varied as a parameter. These comparisons, bear out our inference that for short pulse duration, contribution of single particle effects dominates over that of the collective mechanisms. It is only for long pulses ($>$ 100fs) and large sized cluster that absorption due to collective mechanisms becomes significant. These observations also points to the fact that single particles effects for sufficiently short pulses can in themselves result in as much absorption as due to collective resonance absorption and that one should be careful to attribute any increased absorption of energy in clusters to that due to collective mechanisms alone.

The absorption of energy due to field driven single particle effects were also computed for Ar and was found to follow that due to N$_{2}$ in terms of both number and trend. This can be attributed to the fact that the ionization potentials of Ar and N$_{2}$ are quite close to each other. This, apart from reiterating the significance of single particle processes for short pulse duration(even for a clustered systems like Ar), also helps to provide an experimental way to separate out the contribution to absorption due to collective mechanisms; by simply deducting the absorption values for N$_{2}$ from that of Ar, with all other conditions the same.

These observations were also tested and reproduced in co-linear pump-probe experiments similar to the one described in Fig.\ref{col_pp}.

\subsection{The Evolving Cluster Dielectric}
In the general course of interaction of a cluster with an intense laser field, the initial neutral cluster evolves through a series of steps before it disintegrates into its constituent particles in only a few ps. In this brief time, the largely spherical\cite{shapebogan, xrayscatter, jcphillip86} neutral clusters gets charged, initially, by the photo-ionization processes and subsequently through collisions by the laser driven electrons and charge separation, all the while also being subjected to the internal pressure due to the build-up of heat and electrostatic energy. Thus the clusters in course of these events are subjected to radial internal forces as well as to forces that are directed along the laser polarisation direction\cite{shapeeffectbarrington}. Since the entire evolution of the cluster plasma is dictated by these forces, their own evolution and interplay are, undoubtedly, of fundamental importance. Consequently, any measurable transient property of the cluster nano-plasma that is subject to these forces can be in turn used to comment on these driving forces themselves. One such property is the transient shape of the cluster nano-plasma.

\begin{figure}[h]
  \centering
  \begin{minipage}[!]{0.495\textwidth}
    \includegraphics[width=\textwidth]{col_pp.eps}
    \caption{Co-linear pump-probe experimental setup to study the effect of probe polarization on absorption; the inset potrays the general physical idea behind the experiment(see text).}
    \label{col_pp}
  \end{minipage}
  \hfill
  \begin{minipage}[!]{0.495\textwidth}
    \includegraphics[width=\textwidth]{par_perp.eps}
    \caption{Comparison of probe energy absorption in large Ar clusters with probe polarization either $\parallel$ or $\perp$ to the pump polarization.}
    \label{par_perp}
  \end{minipage}
\end{figure}

Here we describe an experiment to determine this transient shape, the transient ellipticity of the cluster dipole as set up by a plane polarized laser beam. Given the time-frame of cluster evolution, one has to resort to pump-probe measurements to capture the relevant dynamics. For this experiment, we had used the same 3.5 mJ KHz laser beam with pulse duration kept fixed at 25 fs, but, now split into a pump and a probe which were eventually recombined again using a beam-splitter to be focused co-linearly onto the cluster jet. The schematic of the experimental setup is provided in Fig.\ref{col_pp}. The central idea of this experiment was to use a plane polarized pump pulse to set up the cluster plasma and the transient dipole oscillations which in turn affects the absorption of the delayed probe pulse, which is polarized either parallel or perpendicular to the pump and measured by the photo-detector. A half-wave plate was used to rotate the plane of polarization of the probe beam, as required. It was straight forward to determine the absorption of the probe alone, when it was cross polarized with respect to the pump, by simply placing a polarizer set along the direction of the probe polarization, before the detector to cut out the pump from reaching it. But, on the other hand, in the case of the pump and probe being polarized in a mutually parallel direction it was impossible to directly measure the absorption of the probe alone, and hence it had to extracted out of the total absorption of the pump and the probe by using the knowledge of the absorption of the two beams independently while also suitably accounting for their relative strengths. 

The observation of these experiments is presented in Fig.\ref{par_perp} where one can compare the absorption of a plane polarized probe beam as a function of the pump-probe delay $\tau$, when the probe polarization is either parallel or perpendicular to the pump beam. The data presented is for fairly large Ar clusters (average cluster size $\langle N_{c} \rangle$ = 1,22,400) and thus shows the characteristic resonant increase in absorption as the clusters expand in time, as alluded to in the earlier section(s). However, the peak absorption of the probe pulse is about 15\% less for the perpendicularly polarized probe as compared to the one polarized along that of the pump. This apparent disparity can be explained by extending our collective plasma resonance model, which explains how the clusters go through a phase of increased energy deposition, as they expand due to the internal pressures, at some point reaching the optimal density n$_{e}$ $\sim$ n$_{cr}$, where n$_{cr}$ is the critical density for the laser wavelength used. For a spherical dielectric, this optimal density can be worked out to be equal to 3n$_{c}$ because the internal field \textbf{E}$_{in}$, in this case, is related to the field outside \textbf{E}$_{out}$ as
\begin{equation}\label{eqn3}
\textbf{E}_{in}=\frac{3}{|\epsilon+2|}\textbf{E}_{out}.
\end{equation}
However, eqn.\ref{eqn3} has be modified for any change in aspect ratio of the dielectric from its spherical shape. For a ellipsoid dielectric having an aspect ratio of a\(_{x}\) : a\(_{y}\) : a\(_{z}\), this internal field has to modified as

\begin{equation}\label{eqn4}
        {\textbf{E}}_{int}=\sum_{i}\frac{\hat{i}}{1+N_{i}(\epsilon-1)}E_{out_i}
\end{equation}

where the factor

\begin{equation}\label{eqn5}
    N_{i}=\frac{\prod_{x}a_{x}}{2}\int_{0}^{\infty} \frac{1}{(s+a_{i}^{2})\sqrt{\prod_{x}(s+a_{x}^{2})}}ds
\end{equation}

called the depolarization factor is essentially a shape dependent factor that determines the field inside the dielectric\cite{landau}.

Thus, assuming a linearly polarized pump to have set up an ellipsoidal dielectric, we can calculate the energy absorbed from another linearly polarized beam, the probe, as function of the cluster density n$_{e}$, depending on how its polarization is oriented with respect to the shape of this ellipsoidal dielectric. Results of such calculations, for a dielectric do indicate a change in the absorption of the probe depending on its polarization, if the pump is assumed to set up an ellipsoidal dipole that is prolate along its polarization. The experimental data of Fig.\ref{par_perp} can be thus qualitatively explained by the above model and used as the validation of the idea that the overall cluster expansion may indeed be asymmetric as opposed to being perfectly radial. The experimental observations have also been confirmed by particle-in-cell simulations for smaller clusters. In fact, the experimental data could even be used to estimate the ellipticity of the cluster dipoles in the experiment by comparing the ratio experimental absorption values of the parallel and the perpendicular probe beams with those obtained from the calculations for a range of assumed aspect ratios. This value of the aspect ratio was thus estimated to be $\approx$ 1.20 at around the point where the clusters exhibit peak resonant absorption.

Physically however, this difference in absorption is attributed to the the change in the dielectric permittivity $\epsilon$, which can be further argued out to be due to a change in the electron-ion collision frequency $\nu$ which leads to a difference in the electron temperature T$_{e}$ along the axes parallel and perpendicular to the pump polarization.  

\subsection{Generation and Detection of a Poloidal Magnetic Field in Laser-Cluster Interaction}
Intense laser irradiated plasmas has often been probed and proved to be the unique extreme environment where one can observe magnetic fields of enormous strength, regularly reaching MG order\cite{wilks,magsudan}. The major component of this magnetic field, however, is toroidal about the direction of laser propagation and is primarily generated due to the crossed gradients of density and temperature produced in the plasma. The nature and evolution of this toroidal magnetic in laser produced plasmas has been routinely and systematically studied in several experiments employing conventional optical techniques proving their existence and MG strength in the parameters space achievable in femtosecond laser plasma interactions\cite{maggopal,magkahaly,maggourab,magshaikh}.
However the objective of our experiment was to generate and probe an axial (or poloidal; along the laser propagation direction) magnetic field in the cluster plasma by employing the mechanism of Faraday rotation. The basic idea is to study whether a circularly polarized fs pump pulse is capable to generate an axial magnetic field by the mechanism of inverse Faraday rotation, and then to be able to measure its strength by measuring the Faraday rotation of the plane of polarization of a linearly polarized probe pulse propagating coaxially with the pump.

In this experiment we used a 800 nm, 25 fs circularly polarized pump beam focussed onto the cluster jet using a 300 mm lens, potentially producing an axial magnetic field due to the induced circular motion of the electron cloud of the ionized cluster plasma. A delayed plane polarized 400 nm probe beam, focussed using a 500 mm lens(to ensure better overlap) was then made incident onto the interaction region, propagating almost co-linearly but with a small angle ($\lesssim$ 8\textdegree) with the pump propagation direction to be finally detected separately from the pump while also maximising the length of overlap between the two. Before the detector, a high extinction ratio($\sim$10$^6$) Glan-Taylor polarization analyzer  was placed to detect any rotation of the plane of polarization of the probe beam due to Faraday effect, as it propagates through the cluster plasma. However, due to the already demonstrated strong time dependent absorption of the probe, probing the polarization rotation of the probe pulse was not possible with the analyzer axis at 45\textdegree to the initial probe polarization direction, where the sensitivity to rotation is the highest. So the experiment was performed with the axis of the analyzer perpendicular to the initial direction of polarization of the probe, such that any positive signal observed would solely be due to Faraday rotation of the probe polarization. The sensitivity of the Glan-Taylor polarizer-analyzer combination was measured and was found to be sensitive to measure a rotation of the order of 50 mrad. Since, with the given geometry of the pump and probe, an overlap of at least 90 $\mu$m could be expected, which considering a critical density plasma should in turn lead to a Faraday rotation of $\sim$ 53 mrad/T of magnetic field, which with the particular sensitivity of the experimental setup should be possible to detect. Also since a simplistic calculation lead to expecting a B field of $\sim$ 50T/10$^{16}$ W/cm$^{2}$ of intensity of the pump for a critical density plasma\cite{eliezer}, we can expect to be in the parameter space to detect the Faraday rotation of the probe if such simplistic calculation actually hold. The plane polarized probe after passing through the cluster plasma and the crossed Glan-Taylor analyzer was detected on a ccd chip and the mean of the pixel values over and above the background at each delay was observed to measure the rotation of the probe.

\begin{figure}[h]
  \centering
  \begin{minipage}[b]{0.495\textwidth}
    \includegraphics[width=\textwidth]{sensitivity.eps}
    \caption{The mean of the average pixel value over the background as a function of the rotation of plane of polarization of the probe. The experimental setup with the high extinction ratio Glan-Taylor polarizer-analyzer combination allows for a detection of a rotation of $\sim$ 50 mrad of the probe polarization.}
    \label{sensitivity}
  \end{minipage}
  \hfill
  \begin{minipage}[b]{0.495\textwidth}
    \includegraphics[width=\textwidth]{rotation.eps}
    \caption{The mean of the average pixel value over the background as a function of the pump-probe delay. The green line indicates T$_{0}$ and the error bars represents the fluctuations observed over repeated measurement of the mean pixel value. Comparison with Fig.\ref{sensitivity} indicates a probe rotation of $\lesssim$ 45 mrad.}
    \label{rotation}
  \end{minipage}
\end{figure}

The above mentioned quantity is plotted in Fig.\ref{rotation} as a function of the pump-probe delay. Alongside, Fig.\ref{sensitivity} provides the change in the same quantity as function of the angle of rotation of the probe pulse as was obtained during the process of calibrating and checking the sensitivity of the setup. Comparing the two, we find that no significant rotation of the probe was observed and the detected signal indicate an upper-limit to the spatially averaged B field of $\lesssim$ 1 T inside the interaction region which is far less than the MG order toroidal fields generally reported in femtosecond laser plasma interactions.

%\newpage

\bibliography{synopsis_bib}
\bibliographystyle{ieeetr}

\includepdf[pages=-]{list_of_publications.pdf}

\end{document}
